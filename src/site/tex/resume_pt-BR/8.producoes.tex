\cvsection{Produções}

\cvcustomsection{Artigos completos publicados em periódicos}

\begin{cvhonors}
  \cvhonor
    {BARATTO, M.}
    {Multiculturalismo e Direitos Humanos. Conexão, v. V. 3, p. 5-17, 2015.}
    {}
    {1. }
\end{cvhonors}


\cvcustomsection{Capítulo de livros Publicados}

\begin{cvhonors}
  \cvhonor
    {BARATTO, M.}
    {Corte Europeia de Direitos Humanos e o Trabalho. In: Adriana Goulart de Sena Orsini; Flávia de Ávila;Karine Monteiro de Castro Fantini; Nathane Fernandes da Silva. (Org.). Mecanismos de Solução de Controvérsias Trabalhistas nas Dimensões Nacional e Internacional. 1ed.São Paulo: LTr, 2015, v. , p. 261-273.}
    {}
    {1. }
  \cvhonor
    {KOERNER, A. ; BARATTO, M. ; Inatomi, Celly C.}
    {Sobre o Judiciário e a judicialização. In: MOTTA, L. E; MOTA, M.. (Org.). O Estado Democrático de Direito em questão. Rio de Janeiro: Elsevier, 2011, v. , p. -.}
    {}
    {2. }
  \cvhonor
    {Inatomi, Celly C. ; KOERNER, A. ; BARATTO, M.}
    {Judiciário, Reformas e Cidadania no Brasil. In: CUNHA, Alexandre S.;MEDEIROS, Bernardo A.; AQUINO, Luseni M.. (Org.). Estado, Instituições e Democracia: república. 1ed. Brasília: Ipea,2010, v. 1, p. 01-548.}
    {}
    {3. }
\end{cvhonors}

\cvcustomsection{Texto em jornais de notícias/revistas}

\begin{cvhonors}
  \cvhonor
    {BARATTO, M.}
    {Indígenas resistem ao governo federal. Boletim Lua Nova, https://boletimluanova.org/201, 24 abr. 2019.}
    {}
    {1. }
\end{cvhonors}

\cvcustomsection{Trabalhos completos publicados em anais de congressos}

\begin{cvhonors}
  \cvhonor
    {BARATTO, M.}
    {Direitos Indígenas como Direitos Humanos: adesão às regras internacionais e mobilização em decisões judiciais. In: 11 Encontro da ABCP, 2018, Curitiba.}
    {}
    {1. }
  \cvhonor
    {BARATTO, M.}
    {Os direitos, os indígenas e as cortes: a demarcação das terras indígenas no Brasil, Colômbia e Bolívia. In: IX Encontro da ABCP, 2014, Brasília.}
    {}
    {2. }
  \cvhonor
    {BARATTO, M.}
    {Territórios Indígenas e Cortes Constitucionais: uma análise comparada das decisões no Brasil, Colômbia e Bolívia. In: 38 Encontro Anual da ANPOCS, 2014, Caxambu.}
    {}
    {3. }
  \cvhonor
    {BARATTO, M.}
    {Direitos Indígenas na América Latina: o impacto das decisões da OEA nos modelos de cidadania multicultural no Brasil, na Colômbia e na Bolívia. In: 4 Encontro Nacional ABRI, 2013, Belo Horizonte.}
    {}
    {4. }
  \cvhonor
    {BARATTO, M.}
    {A constitucionalização dos direitos indígenas: uma análise comparada. In: 8 Encontro da ABCP, 2012, Gramado - RS. AT09 - Política, Direito e Judiciário, 2012.}
    {}
    {5. }
  \cvhonor
    {BARATTO, M.}
    {Multiculturalismo e Direitos Humanos: uma análise dos paradoxos da diferença. In: IV Seminário Nacional de Ciência Política, 2011, Porto Alegre.}
    {}
    {6. }
  \cvhonor
    {BARATTO, M.}
    {Direitos Indígenas no Brasil, Colômbia e Bolívia: análise dos tribunais constitucionais na contemporaneidade. In: II Seminário Nacional Sociologia\&Política, 2010, Curitiba.}
    {}
    {7. }
  \cvhonor
    {BARATTO, M.}
    {O Debate Islâmico dos Direitos Humanos. In: Encontro Anual da Andhep - Direitos Humanos, Cidades e Desenvolvimento, 2010, Brasília.}
    {}
    {8. }
  \cvhonor
    {BARATTO, M.}
    {Direitos Humanos, Relações de Gênero e Cidadania: Programa Bolsa Família e Microcrédito Indiano em perspectiva comparada. In: 13th BIEN Conference 2010 - Basic Income as an Instrument for justice and peace, 2010, São Paulo.}
    {}
    {9. }
  \cvhonor
    {BARATTO, M.}
    {Quatro propostas de diálogo intercultural: uma análise crítica. In: 34 Encontro anual da Anpocs, 2010, Caxambu, Brasil.}
    {}
    {10. }
  \end{cvhonors}
  \begin{cvhonors}
  \cvhonor
    {BARATTO, M.}
    {The Cross-cultural dialogue and Human Rights. In:21 World Congress of Political Science, 2009, Santiago, Chile}
    {}
    {11. }
  \cvhonor
    {BARATTO, M.}
    {Universalismo versus relativismo Cultural: é possível tornar universais padrões internacionais de direitos humanos?. In: Conferência Internacional Conjunta - International Studies Association (ISA) e Associação Brasileira de Relações Internacionais, ABRI, 2009, Rio de Janeiro.}
    {}
    {12. }
  \cvhonor
    {Koerner, Andrei. ; BARATTO, M. ; Inatomi, Celly C.}
    {Pensamento Jurídico e Decisão Judicial: o processo de controle
    concentrado em decisões do Supremo Tribunal Federal pós-1988. In: 31o Encontro Anual da ANPOCS, 2007, Caxambu-MG.}
    {}
    {13. }
  \cvhonor
    {BARATTO, M.}
    {Diálogo Intercultural: as possibilidade de legitimidade e eficácia dos direitos humanos em perspectiva trans-cultural. In: 1o Simpósio de Pós-Graduação em Relações Internacionais do Programa Santiago Dantas, 2007, São Paulo.}
    {}
    {14. }
\end{cvhonors}

\cvcustomsection{Resumo expandidos publicados em anais de congressos}

\begin{cvhonors}
  \cvhonor
    {BARATTO, M.}
    {Quatro propostas de diálogo intercultural: uma análise crítica. In: 34 Encontro da Anpocs, 2010, Caxambu.}
    {}
    {1. }
  \cvhonor
    {BARATTO, M.}
  {Estado Democrático de Direito e Direitos Humanos: uma relação necessária. In: II Jornada de Filosofia e Direitos Humanos, 2006, Londrina.}
    {}
    {2. }
\end{cvhonors}

\cvcustomsection{Resumo publicados em anais de congressos}

\begin{cvhonors}
  \cvhonor
    {BARATTO, M.}
    {Cidadania, direitos Indígenas e cortes constitucionais: conflitos sobre a demarcação das terras indígenas no Brasil, Colômbia e Bolívia. In: Primero Congresso Internacional Los Pueblos Indígenas de América Lainta, siglos XIX-XXI, 2013, Oaxaca.}
    {}
    {1. }
  \cvhonor
    {BARATTO, M.}
    {Direitos Indígenas na América do Sul: constitucionalismo e direitos humanos em perspectiva multicultural. In: II Colóquio Internacional de Direitos Humanos, 2011, Rio de Janeiro.}
    {}
    {2. }
  \cvhonor
    {BARATTO, M.}
    {Human Rights and cross-cultural dialogue. In: Second Global International Studies Conference, 2008, Ljubljana.}
    {}
    {3. }
\end{cvhonors}

\cvcustomsection{Apresentações de Trabalho}

\begin{cvhonors}
  \cvhonor
    {BARATTO, M.}
    {Direitos Indígenas positivados: Avanços e Desafios Contemporâneos. 2013. (Apresentação de Trabalho/Conferência ou palestra).}
    {}
    {1. }
  \cvhonor
    {BARATTO, M.; REGO, W. L. ; BAPTISTA, L.}
    {Direitos Humanos, Relações de Gênero e Cidadania: Programa Bolsa Família e Microcrédito Indiano em Perspectiva Comparada. 2010. (Apresentação de Trabalho/Congresso).}
    {}
    {2. }
  \cvhonor
    {BARATTO, M.}
    {Diálogo Intercultural e seus críticos: possibilidades e limites. 2008. (Apresentação de Trabalho/Seminário).}
    {}
    {3. }
  \cvhonor
    {BARATTO, M.}
    {Diálogo intercultural: uma análise crítica. 2008. (Apresentação de Trabalho/Comunicação).}
    {}
    {4. }
  \cvhonor
    {BARATTO, M.}
    {Direitos Humanos e Diálogo Intercultural. 2008. (Apresentação de Trabalho/Seminário).}
    {}
    {5. }
  \cvhonor
    {BARATTO, M.}
    {Diálogo intercultural: as possibilidades de legitimidade e eficácia dos Direitos Humanos em perspectiva trans-localista. 2007. (Apresentação de Trabalho/Comunicação).}
    {}
    {6. }
  \cvhonor
    {BARATTO, M.}
    {Estado Democrático de Direito e Direitos Humanos: uma relação necessária. 2006. (Apresentação de Trabalho/Outra).}
    {}
    {7. }
  \cvhonor
    {BARATTO, M.}
    {Direitos Humanos na atualidade: principais problemas teóricos e dilemas práticos. 2006. (Apresentação de Trabalho/Conferência ou palestra).}
    {}
    {8. }
\end{cvhonors}

\cvcustomsection{Demais tipos de produção técnica}

\begin{cvhonors}
  \cvhonor
    {BARATTO, M.}
    {Direitos indígenas e teoria política: os processos recentes de constitucionalização dos direitos indígenas. 2017. (Relatório de pesquisa).}
    {}
    {1. }
  \cvhonor
    {BARATTO, M.; Bazzano, A.}
    {PRODUÇÃO BIBLIOGRÁFIA SOBRE DITADURA E GÊNERO NO BRASIL. 2013. (Relatório de pesquisa).}
    {}
    {2. }
\end{cvhonors}
\begin{cvhonors}
  \cvhonor
    {BARATTO, M.}
    {Direitos humanos e abordagens de pesquisa: iniciação à leitura do campo. 2011. (Curso de curta duração ministrado/Extensão).}
    {}
    {3. }
  \cvhonor
    {BARATTO, M.; Bazzano, A.}
    {Direitos Humanos, Movimentos Sociais e a OEA. 2011. (Relatório de pesquisa).}
    {}
    {4. }
  \cvhonor
    {BARATTO, M.}
    {Decisões arbitrais da Corte Islâmica Inglesa: um estudo de caso sobre instituições multiculturais. 2010. (Relatório de pesquisa).}
    {}
    {5. }
  \cvhonor
    {KRITSCH, R. ; BARATTO, M.}
    {Reunião de Trabalho do Grupos de Estudos em Teoria Política. 2009. (Reunião de Grupo de Trabalho).}
    {}
    {6. }
  \cvhonor
    {BARATTO, M.; BAPTISTA, L. ; REGO, W. L.}
    {Relatório de atividades do projeto: Direitos humanos, relações de gênero e política social: análise da recepção de programas de renda focados na mulher na Índia e no Brasil. 2008. (Relatório de pesquisa).}
    {}
    {7. }
  \cvhonor
    {BARATTO, M.}
    {Direitos Humanos e Multiculturalismo. 2008. (Relatório de pesquisa).}
    {}
    {8. }
  \cvhonor
    {BARATTO, M.}
    {Análise das decisões liminares em ADIN no período de 1988 à 1994. 2008. (Relatório de pesquisa).}
    {}
    {9. }
\end{cvhonors}

\cvcustomsection{Demais Trabalhos}

\begin{cvhonors}
  \cvhonor
    {BARATTO, M.}
    {Diálogo Intercultural e Direitos Humanos: possibilidades e limites. 2009 (Dissertação de Mestrado).}
    {}
    {1. }
  \cvhonor
    {BARATTO, M.}
    {Estado, Democracia e Direitos Humanos: uma visão Trans-localista. 2005 (Trabalho de Conclusão de Curso).}
    {}
    {2. }
  \cvhonor
    {KRITSCH, R.; BARATTO, M.}
    {Direitos humanos, Estado e soberania: alguns problemas teóricos e práticos. 2003
    (Atvidade Acadêmica Complementar).}
    {}
    {3. }
\end{cvhonors}
