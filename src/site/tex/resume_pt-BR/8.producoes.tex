\cvsection{Produções}

\cvcustomsection{Artigos completos publicados em periódicos}

\begin{cvhonors}
  \cvhonor
    {BARATTO, M.}
    {Multiculturalismo e Direitos Humanos. Conexão, v. V. 3, p. 5-17, 2015.}
    {}
    {1. }
\end{cvhonors}


\cvcustomsection{Capítulo de livros Publicados}

\begin{cvhonors}
  \cvhonor
    {BARATTO, M.}
    {Corte Europeia de Direitos Humanos e o Trabalho. In: Adriana Goulart de Sena Orsini; Flávia de Ávila;Karine Monteiro de Castro Fantini; Nathane Fernandes da Silva. (Org.). Mecanismos de Solução de Controvérsias Trabalhistas nas Dimensões Nacional e Internacional. 1ed.São Paulo: LTr, 2015, v. , p. 261-273.}
    {}
    {1. }
  \cvhonor
    {KOERNER, A. ; BARATTO, M. ; Inatomi, Celly C.}
    {Sobre o Judiciário e a judicialização. In: MOTTA, L. E; MOTA, M.. (Org.).
   O Estado Democrático de Direito em questão. Rio de Janeiro: Elsevier, 2011, v. , p. -.}
    {}
    {2. }
  \cvhonor
    {Inatomi, Celly C. ; KOERNER, A. ; BARATTO, M.}
    {Judiciário, Reformas e Cidadania no Brasil. In: CUNHA, Alexandre S.;MEDEIROS, Bernardo A.; AQUINO, Luseni M.. (Org.). Estado, Instituições e Democracia: república. 1ed.Brasília: Ipea,2010, v. 1, p. 01-548.}
    {}
    {3. }
\end{cvhonors}

\cvcustomsection{Texto em jornais de notícias/revistas}

\begin{cvhonors}
  \cvhonor
    {BARATTO, M.}
    {Indígenas resistem ao governo federal. Boletim Lua Nova, https://boletimluanova.org/201, 24 abr. 2019.}
    {}
    {1. }
\end{cvhonors}
\pagebreak
\cvcustomsection{Trabalhos completos publicados em anais de congressos}

\begin{cvhonors}
  \cvhonor
    {BARATTO, M.}
    {Direitos Indígenas como Direitos Humanos: adesão às regras internacionais e mobilização em decisões judiciais. In: 11 Encontro da ABCP, 2018, Curitiva. Anais do 11° Encontro da ABCP, 2018.}
    {}
    {1. }
  \cvhonor
    {BARATTO, M.}
    {Os direitos, os indígenas e as cortes: a demarcação das terras indígenas no Brasil, Colômbia e Bolívia. In: IX ENCONTRO DA ABCP, 2014, Brasília. IX ENCONTRO DA ABCP, 2014.}
    {}
    {2. }
  \cvhonor
    {BARATTO, M.}
    {Territórios Indígenas e Cortes Constitucionais: uma análise comparada das decisões no Brasil, Colômbia e Bolívia. In: 38 Encontro Anual da ANPOCS, 2014, Caxambu. Anais do 38 Encontro Anual da ANPOCS, 2014.}
    {}
    {3. }
  \cvhonor
    {BARATTO, M.}
    {Direitos Indígenas na América Latina: o impacto das decisões da OEA nos modelos de cidadania multicultural no Brasil, na Colômbia e na Bolívia. In: 4 ENCONTRO NACIONAL ABRI, 2013, Belo Horizonte. Anais do 4 Encontro Nacional Abri, 2013.}
    {}
    {4. }
  \cvhonor
    {BARATTO, M.}
    {A constitucionalização dos direitos indígenas: uma análise comparada. In: 8 Encontro da ABCP, 2012, Gramada - RS. AT09 - Política, Direito e Judiciário, 2012.}
    {}
    {5. }
  \cvhonor
    {BARATTO, M.}
    {Multiculturalismo e Direitos Humanos: uma análise dos paradoxos da diferença. In: IV Seminário Nacional de Ciência Política, 2011, Porto Alegre. Anais - IV Seminário Nacional de Ciência Política, 2011.}
    {}
    {6. }
  \cvhonor
    {BARATTO, M.}
    {Direitos Indígenas no Brasil, Colômbia e Bolívia: análise dos tribunais constitucionais na contemporaneidade. In: II Seminário Nacional Sociologia\&Política, 2010, Curitiba. II Seminário Nacional Sociologia\&Política, 2010. v. 01.}
    {}
    {7. }
  \cvhonor
    {BARATTO, M.}
    {O Debate Islâmico dos Direitos Humanos. In: Encontro Anual da Andhep - Direitos Humanos, Cidades e Desenvolvimento, 2010, Brasília. Anais do 10 Encontro da Andhpe, 2010.}
    {}
    {8. }
  \cvhonor
    {BARATTO, M.}
    {Direitos Humanos, Relações de Gênero e Cidadania: Programa Bolsa Família e Microcrédito Indiano em perspectiva comparada. In: 13th BIEN CONGRESS 2010 Basic Income as an Instrument for justice and peace, 2010, São Paulo. Programme, 2010.}
    {}
    {9. }
  \cvhonor
    {BARATTO, M.}
    {Quatro propostas de diálogo intercultural: uma análise crítica. In: 34 Encontro anula da Anpocs, 2010, Caxambu. Anais do 34o. Encontro anual da Anpocs, de 25 a 29 de outubro de 2010, em Caxambu/MG, 2010.}
    {}
    {10. }
  \cvhonor
    {BARATTO, M.}
    {The Cross-cultural dialogue and Human Rights. In:21 World Congress of Political Science, 2009, Santiago. 21 World Congress of Political Science, 2009.}
    {}
    {11. }
  \cvhonor
    {BARATTO, M.}
    {Universalismo versus relativismo Cultural: é possível tornar universais padrões internacionais de direitos humanos?. In: CONFERÊNCIA INTERNACIONAL CONJUNTA -INTERNATIONAL STUDIES ASSOCIATION - ISA E ASSOCIAÇÃO BRASILEIRA DE RELAÇÕES INTERNACIONAIS ? ABRI, 2009, Rio de Janeiro. CONFERÊNCIA INTERNACIONAL CONJUNTA -INTERNATIONAL STUDIES ASSOCIATION - ISA E ASSOCIAÇÃO BRASILEIRA DE RELAÇÕES INTERNACIONAIS ? ABRI, 2009.}
    {}
    {12. }
  \cvhonor
    {Koerner, Andrei. ; BARATTO, M. ; Inatomi, Celly C.}
    {Pensamento Jurídico e Decisão Judicial: o processo de controle
    concentrado em decisões do Supremo Tribunal Federal pós-1988. In: 31o Encontro Anual da ANPOCS, 2007, Caxambu,
    MG. 31o Encontro Anual da ANPOCS, 2007.}
    {}
    {13. }
  \cvhonor
    {BARATTO, M.}
    {DIÁLOGO INTERCULTURAL: as possibilidade de legitimidade e eficácia dos direitos humanos em perspectiva trans-cultural. In: 1o Simpósio de Pós-Graduação em Relações Internacionais do Programa Santiago Dantas, 2007, São Paulo. 1o Simpósio de Pós-Graduação em Relações Internacionais do Programa Santiago Dantas, 2007.}
    {}
    {14. }
\end{cvhonors}
