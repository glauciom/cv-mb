\cvsection{Projetos de Pesquisa}
\begin{cvhonors}
  \cvhonor
    {Universidade Estadual de Campinas}
    {\textbf{Mudança institucional e pensamento constitucional: uma análise da reflexão jurídica sobre controles judiciais da autoridade política no Brasil (1920-1988)}}
    {Campinas, SP}
    {2012-2017}
\end{cvhonors}
\begin{cvhonors}
  \cvhonor
    {Universidade Estadual de Campinas}
    {\textbf{O Pensamento Jurídico Publicista brasileiro transformações e vertentes (1920-1960)}}
    {Campinas, SP}
    {2012-2015}
\end{cvhonors}
\begin{cvhonors}
  \cvhonor
    {Universidade Estadual de Londrina}
    {\textbf{Para além da constelação nacional? Disputas em torno da cidadania, do cosmopolitismo e dos direitos humanos na teoria política contemporânea}. Esta investigação (aprovada pela Chamada Pública PQ-2011,processo no 307085/2011-2, do CNPq) visa a dar continuidade ao projeto de pesquisa Direitos humanos universais e Estados nacionais: fundamentos históricos e problemas teóricos II , encerrado em Fevereiro/2012. No projeto anterior, constatou-se a interdependência entre cidadania nacional, Estado, direitos humanos e democracia, movimento que impôs a necessidade de se averiguar as possibilidades de se pensar os direitos humanos fora do quadro das estruturas institucionais e de poder atualmente vigentes bem como avaliar alternativas factíveis a tal interdependência. Para levar a bom termo tal tarefa, o caminho escolhido foi o de aprofundar o estudo do tema do cosmopolitismo e sua relação com os direitos humanos e com a cidadania de base nacional elementos que guardam uma complexa, e por vezes até mesmo contraditória, interação com propostas e ideais cosmopolitas. Assim, por meio do estudo das muitas teorias da cidadania cosmopolita, que vêm ganhando adeptos cada vez mais numerosos nas mais distintas vertentes do pensamento político, pretende-se averiguar as propostas de universalização dos direitos humanos e da cidadania a partir de contextos nacionais bem como as condições de execução deste ideário cosmopolita diante do atual cenário internacional mundial que, segundo algumas vozes sonoras, caminharia na direção de uma constelação pós-nacional, embora as práticas políticas das principais potências hegemônicas globais venham apontando acentuado recrudescimento das possibilidades de extensão da cidadania a não-nacionais.}
    {Londrina, PR}
    {2012-2016}
\end{cvhonors}
\begin{cvhonors}
  \cvhonor
    {Universidade Estadual de Londrina}
    {\textbf{Direitos humanos, cosmopolitismo, cidadania e teoria política: questões teóricas e problemas práticos}. Este projeto, aprovado pela Fundação Araucária (Chamada Pública 14/2009 Ato 097/2010), visa a dar continuidade ao projeto de pesquisa Direitos humanos, Estado e soberania: alguns problemas teóricos e práticos II , encerrado em Maio/2011, financiado pela mesma Fundação Araucária. O discurso em prol da implementação dos direitos humanos em âmbito global tornou-se, no mundo ocidental moderno, lugar comum e matéria inquestionável. As pressuposições ontológicas da idéia de direitos humanos bem como a pretensão de que o poder político deve estar sujeito às exigências da razão e da lei passaram a fazer parte, no mundo moderno, de uma constituição política e jurídica que encontra na figura do Estado nacional uma de suas mais fortes expressões, o que, com freqüência, opõe a natureza particularista dos direitos do homem e do cidadão ao caráter universalista das cartas de direitos humanos. A discussão sobre os direitos humanos, com especial ênfase na possibilidade de sua universalização a partir de contextos nacionais, isto é, com ênfase numa abordagem cosmopolita, encontra no projeto habermasiano (e de seus seguidores) de constituição de um Estado mundial (Weltstaat), fundado num patriotismo constitucional à Carta de Direitos Humanos da ONU, um de seus mais fortes sustentáculos teóricos. De outro lado, paradoxos e tensões foram potencializados ao longo da segunda metade do século XX, entre outras coisas, pelo acirramento do processo de globalização, que vem acarretando mudanças, inclusive, no conceito de cidadania democrática. Ora, questões de peso para a teoria política se colocam também aqui: Como tais mudanças devem ser analisadas? Estamos de fato diante de uma nova modalidade de cidadania? Os Estados nacionais de base territorial estão realmente perdendo força? Tais são as questões a se aprofundar nesta pesquisa.}
    {Londrina, PR}
    {2011-2013}
\end{cvhonors}
\begin{cvhonors}
  \cvhonor
    {Universidade Estadual de Londrina}
    {\textbf{Direitos humanos universais e Estados nacionais: fundamentos históricos e problemas teóricos II}. O objetivo desta investigação é dar continuidade à pesquisa de mesmo nome iniciada no estágio pós-doutoral realizado em 2007-2008. O projeto pretende examinar de maneira mais aprofundada os fundamentos históricos e teóricos das questões e transformações que envolvem a noção de direitos humanos e suas relações com noções fundamentais da Teoria Política bem como suas implicações para as idéias de Estado soberano, cidadania, direito internacional e democracia. Em particular, pretende-se aprofundar a temática da tensão entre tendências universalistas e particularismos locais um tema visto por inúmeros especialistas como uma falsa questão. Para tanto, será necessário dar continuidade ao estudo dos modelos de análise dos direitos humanos disponíveis na literatura de área, de modo a permitir a comparação de paradigmas e abordagens teóricas existentes, como por exemplo a visão de sociedade e do sujeito de direito em vertentes do pensamento político contemporâneo tais como o republicanismo, o liberalismo, o comunitarismo, a teoria discursiva, etc. Tal investigação permitirá, entre ouras coisas, averiguar e avaliar as possibilidades de pensar os direitos humanos fora do quadro das estruturas institucionais e de poder atualmente vigentes bem como avaliar alternativas factíveis à interdependência entre cidadania, Estado e direitos humanos, tal como verificados até aqui.}
    {Londrina, PR}
    {2009-2012}
\end{cvhonors}
\begin{cvhonors}
  \cvhonor
    {Universidade Estadual de Campinas}
    {\textbf{Direitos humanos, relações de gênero e política social: análise da recepção de programas de renda focados na mulher na Índia e no Brasil}. A pesquisa trata dos efeitos sociais e político-morais de programas de renda adotados nos últimos anos no Brasil e na Índia que atribuem a titularidade das prestações às mulheres, tendo em vista os motivos explicitados pelos formuladores dos programas para atribuir-lhes as prestações e verificar as conseqüências dessas políticas para a construção de sua identidade, bem como examinar as relações sociais e as de gênero, no interior do complexo das relações sociais locais em que elas se inserem. O projeto se justifica pela constatação de tensões no ordenamento das relações de gênero no interior da família ampliada e das relações sociais locais, associadas aos referidos programas. A análise dos efeitos de programas análogos em contextos culturais diferentes, como comunidades rurais do Brasil e Índia, permitirá identificar temas, dimensões analíticas e conceitos relevantes para a pesquisa sobre efeitos de políticas de transferência de rendas focadas nas mulheres no Brasil. O objetivo geral é elaborar um quadro conceitual e hipóteses referentes, por um lado, às diferenças de recepção dos recursos, pelos destinatários, das políticas de distribuição de renda focadas na mulher no Brasil e na Índia e colhendo subsídios teóricos e empirícos a continuidade à investigação sobre os efeitos políticos e sociais do Programa-Bolsa Família. A pesquisa pressupõe que os destinatários de políticas de promoção de direitos são considerados como participantes ativos e socialmente integrados nas diversas sociedades locais. A questão da recepção das políticas é articulada com a problemática teórica da promoção dos direitos humanos em context de grande desigualdade social e diversidade cultural. Outra premissa é que a recepção das mecionadas políticas se articulada com a problemática teórica da promoção dos direitos humanos em contexto de grande diversidade social e cultural. Os resultados esperados são: a revisão bibliográfica sobre os impactos de políticas de renda mínima centradas.}
    {Campinas, SP}
    {2008-2009}
\end{honors}
\end{cvhonors}
\begin{cvhonors}
  \cvhonor
    {Universidade Estadual de Campinas}
    {\textbf{Efetivação dos direitos humanos e política social: uma análise da recepção do Programa Bolsa Família no Brasil em perspectiva comparada com programas de rendas na India}. A pesquisa trata de um aspecto dos programas de distribuição de renda (micro-crédito e bolsa-família) adotados nos últimos anos pelo governo federal brasileiro: os efeitos da atribuição, por esses programas, da titularidade das prestações e dos créditos às mulheres. A pesquisa aborda, por um lado, os motivos explicitados pelos formuladores das políticas para centrar o programa nas mulheres e, por outro lado, as conseqüências desse enfoque para a identidade das mulheres, as suas relações sociais e de gênero e o complexo das relações sociais locais. Do ponto de vista dos formuladores, vê-se que a atribuição da titularidade às mulheres dos créditos ou benefícios baseia-se numa conjugação de objetivos, que estão muitas vezes em tensão entre si: a maior efetividade da destinação do recurso, a promoção do desenvolvimento local, o empowerment especifico das mulheres, com mudanças em seu status sócio-econômico, oportunidades de participação e relações de gênero. Do ponto de vista do impacto para os destinatários, as mulheres, nota-se que a titularidade das prestações, além de elevar os recursos econômicos à sua disposição das mulheres, provoca importantes mudanças no seu status sócio-econômico e familiar, envolvendo sua auto-estima, sua maior participação nas decisões domésticas e nas atividades que se estendem para além do círculo familiar, como as mercantis. Ao mesmo tempo, notou-se que, além dessas melhorias na condição das mulheres, vê-se manifestações de ressentimento, por parte de seus companheiros e da parcela masculina das comunidades locais, reveladores de um sentimento de humilhação social, em virtude de sua incapacidade de prover quaisquer recursos monetários para a manutenção da família e da sua exclusão das decisões relevantes sobre o orçamento doméstico. Essas observações indicam tensões no ordenamento das relações sociais e a precariedade das mudanças iniciadas pelo programa.}
    {Campinas, SP}
    {2008-2009}
\end{honors}
\end{cvhonors}
\begin{cvhonors}
  \cvhonor
    {Universidade Estadual de Londrina}
    {\textbf{Direitos Humanos, Estado e soberania: alguns problemas teóricos e práticos II}. O objetivo desta investigação é dar continuidade ao projeto de mesmo nome desenvolvido entre 2003-2006, também financiado pela Fundação Araucária. A partir do trabalho já desenvolvido até aqui, pretende-se examinar de maneira mais aprofundada os fundamentos históricos e teóricos das questões e transformações que envolvem a noção de direitos humanos, e especialmente suas implicações para as idéias de Estado soberano (em particular, na forma do Estado constitucional), de cidadania, de direito internacional e de democracia, regime que hoje contempla, entre outras coisas, o respeito à diversidade histórica e cultural dos povos e dos grupos humanos. Em especial, pretende-se com esta pesquisa aprofundar o problema da tensão entre tendências universalistas e particularismos locais, tanto no que respeita à implementação de direitos humanos universais em contextos específicos quanto no que se refere à suposta crise dos Estados nacionais diante da internacionalização econômica, social, cultural e política. O estudo das questões acima mencionadas permitirá, entre outras coisas, uma compreensão mais adequada dos acontecimentos recentes que vêm alterando com enorme rapidez a cena política contemporânea, como os atentados de 11 de Setembro nos EUA e suas conseqüências, entre as quais merece destaque a doutrina de intervenção preventiva formulada pelos republicanos norte-americanos e levada a cabo pelos EUA e seus aliados no Afeganistão e no Iraque. Tais transformações impõem ainda uma análise e avaliação das novas formas institucionais emergentes bem como dos novos movimentos sociais, como a reivindicação de alguns grupos em defesa do estabelecimento de uma espécie de sociedade civil global, na qual temas transversais (como a preservação da água no planeta) venham a ser objeto de discussão e deliberação por todos os cidadãos do mundo, independentemente de sua filiação a este ou aquele Estado nacional.}
    {Londrina, PR}
    {2007-2011}
\end{honors}
\end{cvhonors}
\begin{cvhonors}
  \cvhonor
    {Universidade Estadual de Londrina}
    {\textbf{Formação e Pesquisa em Teoria Política - Grupo Estudos em Teoria Política - CNPq}. Financiamento de pesquisa obtido por meio da participação no Edital FAEPE/UEL 002/2007, \"Modalidade C\" (Grupos de Pesquisa). O financiamento desta pesquisa tem como objetivo fomentar o desenvolvimento da pesquisa, ensino e extensão no âmbito institucional, mediante o apoio financeiro a propostas de pesquisa, ensino e extensão, aprovadas e divulgadas por meio do Edital FAEPE/UEL no 04/2006. No caso específico do Getepol-CNPq, tais pesquisas estão sendo levadas a cabo em forma de projetos de Mestrado, iniciação científica bem como TCCs e orientações de outra natureza, em duas linhas de pesquisa: 1) Direitos humanos, políticas públicas e relações internacionais; e 2) Estado, soberania e outros temas de teoria política.}
    {Londrina, PR}
    {2007-2008}
\end{honors}
\end{cvhonors}
\begin{cvhonors}
  \cvhonor
    {Universidade Estadual de Campinas}
    {\textbf{Estados Unidos: Impactos de suas políticas para a reconfiguração do sistema internacional}. Pesquisa sobre os impactos da atuação dos Estados Unidos, centrada em três domínios:  ́política comercial, estruturas de governança internacional, e segurança e percepões de conflitos. A pesquisa reúne docentes do Programa San Tiago Dantas de Relações Internacionais (Unicamp, Unesp e PUC), e é realizado no Cedec.}
    {Campinas, SP}
    {2007-2010}
\end{honors}
\end{cvhonors}
\begin{cvhonors}
  \cvhonor
    {Universidade Estadual de Campinas}
    {\textbf{Pensamento Constitucional do STF nos anos noventa}. A presente pesquisa propõe realizar uma análise política do controle judicial da constitucionalidade das leis e atos normativos realizado pelo Supremo Tribunal Federal nos anos noventa. A pesquisa aborda, inicialmente, o Tribunal em função de suas características institucionais e do contexto político de reforma do Estado brasileiro nos anos noventa. Define-se um quadro conceitual para a análise das decisões judiciais segundo três dimensões: normativa, estratégica e cognitiva. Essas dimensões são utilizadas para, a partir da análise de séries de decisões, caracterizar regimes jurisprudenciais definidos pelo Tribunal em temas selecionados. Esses temas referem-se ao caráter da Constituição, ao mandato do Tribunal, as características das ações judiciais de controle da constitucionalidade, e a compatibilidade entre a reforma do Estado e direitos fundamentais. O objetivo central da pesquisa é, com a caracterização e comparação dos regimes jurisprudenciais, identificar o pensamento constitucional elaborado pela maioria do ministros do STF, assim como as diferentes vertentes e eventuais mudanças de orientação ao longo da década. A pesquisa propõe contribuir para explicar o padrão de atuação do STF no período, situando as decisões nas interações estratégicas entre ministros e agentes políticos, e das estratégias de mobilização legal de atores coletivos.}
    {Campinas, SP}
    {2005-2010}
\end{honors}
\end{cvhonors}
\begin{cvhonors}
  \cvhonor
    {Universidade Estadual de Londrina}
    {\textbf{Direitos Humanos, Estado e soberania: alguns problemas teóricos e práticos}. O objetivo geral deste projeto é empreender uma abordagem sistemática, do ponto de vista da teoria política, das questões e dos dilemas envolvidos na problemática dos direitos humanos. Como é possível falar, p. ex., em direitos humanos universais, válidos para todos os seres humanos, e, ao mesmo tempo, respeitar as particularidades de cada sociedade e/ou de cada povo, etnia ou grupo específico? Ou ainda: será possível o convívio de formas institucionais aparentemente excludentes como Estados nacionais soberanos e organizações de caráter político inter ou supranacionais, como a ONU, as OSCs ou as ONGs, algumas delas também com pretensões de emitir comandos e declarações de direitos vinculantes? Tais questões não parecem ter resposta fácil: sem noções como as de indivíduo, de Estado ou de ordenamento legal, p. ex., não há sequer a possibilidade de se enunciar a defesa de direitos humanos tais como pensados contemporaneamente. Isto é, sem estruturas constitucionais específicas, garantidas em última instância pela força centralizada nos Estados nacionais, não há como pensar comandos vinculantes eficazes capazes de garantir tais \"direitos\". A presença do chamado Estado de Direito é, portanto, fundamental. Também a teoria política se vê, neste ponto, diante de novas questões complexas: qual seria o arranjo político e institucional mais adequado para a promoção de um ideário como o dos direitos humanos? Uma ordem política centralizada supranacionalmente, uma federação de repúblicas autônomas (como quer a União Européia) ou um novo pacto entre Estados? Mudanças de rumo no debate teórico como essas aqui brevemente enunciadas nos mostram que a dinâmica das relações nas quais se insere a problemática dos direitos humanos está em processo de transformação e requer, por isto, um esforço de sistematização, a fim de que se possa chegar a uma melhor compreensão conceitual das questões em jogo. Localizar e situar tais mudanças será a tarefa central deste trabalho.}
    {Londrina, PR}
    {2003-2006}
\end{honors}
\end{cvhonors}
\begin{cvhonors}
  \cvhonor
    {Universidade Estadual de Londrina}
    {\textbf{Universalismo, particularismo e direitos humanos: origens históricas e problemas teóricos}. Diante dos inúmeros acréscimos que ganhou a temática dos direitos humanos nos últimos anos, cresceu também o grau de especialização e de refinamento conceitual no trato dos temas que a envolvem. Poucos estudos, no entanto, têm tratado das raízes históricas e da construção dos vários discursos a respeito dos direitos humanos. Esse percurso nos permitiria entender melhor a formação do conceito. A idéia de estudar tal matéria conduziu-nos ao período de consolidação da noção de um homem universal portador de direitos, fundamento e premissa da noção de direitos humanos. Esta escolha nos levou aos pensadores do final do século XVIII. Para mostrar o estado da questão no período - que se traduz modernamente no debate entre universalismo e relativismo -, escolheu-se dois autores paradigmáticos: o filósofo alemão Immanuel Kant, que fez provavelmente a mais contundente defesa da construção de uma comunidade do gênero humano, fundada na noção de um direito humano universal; e seu contemporâneo e opositor teórico, o político e filósofo conservador inglês e severo crítico do Iluminismo em qualquer das suas vertentes, Edmund Burke. Refazer este percurso histórico-conceitual é o empreendimento aqui proposto.}
    {Londrina, PR}
    {2002-2008}
\end{honors}
\end{cvhonors}
\pagebreak
